\subsection*{Strong Scaling Study}

An important finding of our strong scaling study was that speedup was consistently
the greatest when using 4 pthreads per MPI rank, and overall usign a moderate number of
threads for each rank outperformed the more extreme configurations. This result
means that using pthreads in addition to MPI ranks produces both positive and
negative performance tradeoffs.

Figure \ref{fig:comm} shows that using more pthreads per MPI rank generally
reduces the time spent for communication. This is likely what produces the performance
increase when using multiple pthreads per rank. Initially, when using a small
number of parallel processes (less than 512) communication makes up very little
of the total time required to run the simulation; however as we scale up the
number of parallel processes while holding the size of the simulation constant,
we observe that communication quickly becomes a major contributor to total
execution time (Figure \ref{fig:commpercent}). This relationship was likely the
cause for the total speedup behavior of the single thread per rank case (Figure
\ref{fig:totalspeedup}), which had much worse performance than all other cases as we
scaled to large numbers of parallel processes.

The slower communication times for the cases with fewer threads per rank is
likely due to there being more MPI ranks involved in the collective MPI\_Allgather
operations than in cases with more threads per rank. Specifically, the single
thread case requires 64 MPI ranks per compute node to be involved in the collective
message passing, while the case with 64 ranks requires only a single MPI rank per
node. The total volume of data being passed between ranks is still the same
(since each rank needs to gather the entire array of birds in the universe),
however, when more threads are used fewer, though larger, messages are passed.
Although the Blue Gene/Q has a very efficient and specialized network that allows
these messages to be passed quickly, it is possible that small messages are not
passed as efficiently as large messages due to memory block sizes. That, combined
with the larger volume of small messages that need to be passed for low thread
configurations, likely created a decrease in communication performance that
becomes especially apparent for large scale parallel computation.

There must exist some limitation for pthreads however, since the configurations
with many threads per rank were outperformed by other configurations with more
moderate numbers of threads (Figure \ref{fig:totalspeedup}). This limiting factor
appears to be the compute time (Figure \ref{fig:compute}). Updates for each
bird are computed individually and written to separate variables that do not
get read by threads that are updating other birds. These separate variables
comprise a ``future'' state for that bird, and once all of the updates have
been completed across all threads, then they are applied to each bird. This
approach doesn't require shared memory to be locked while the updates are happening,
since reads and writes occur in separate places. While this overcomes some of the
performance issues inherent with shared memory implementations like pthreads, we
still are required to use a pthread barrier to wait for all threads to finish
updating before applying the changes in shared memory. This means that each rank
is limited in performance to its slowest pthread. When the number of pthreads per
MPI rank is low, this bottleneck is not very significant, however as the number
of threads increase this additional barrier can degrade performance. Further,
using more threads on each BG/Q compute node decreases the locality of the shared
memory that each thread needs to access, which could also decrease performance.
Cases that utilize many pthreads still spend a substantial amount of their total
execution time in computation (Figure \ref{fig:commpercent}), therefore this
slowdown caused by many pthreads can limit any performance gains made by faster
MPI communications.

\subsection*{Weak Scaling Study}

After our strong scaling study, we selected the configuration with the best
performance, being 4 pthreads per rank and 16 MPI ranks per BG/Q node, to
perform a weak scaling study where we increase the number of birds proportionally
to the number of parallel processes we are using. We decided to use 1024 birds
per BG/Q node, which means that each individual thread was responsible for 16
birds. We ran this simulation for 1000 iterations using 1, 2, 4, 8, 16, 32, 64,
and 128 nodes. The maximum number of birds that we simulated was 131,072 (using
128 nodes).

The time required to run each of these simulations roughly increased proportionally
to the number of birds that we were simulating (Figure \ref{fig:weakscaling}).
Each bird has to compare itself with every other bird in the simulation, so that
the running time of the update computations were \(O(n^2)\) with respect to the
number of birds. Therefore, a linear increase in the number of compute nodes
relative to the number of birds being simulated was not able to maintain a
constant performance.

The maximum number of events per second that we acheived was 248,684.2 which
occured while using 2048 parallel processes, or 32 BG/Q compute nodes (Figure
\ref{fig:eventspersec}). The events/second actually decreased when using 64 and
128 nodes. This was likely due to the computationally intensive update algorithm
and sublinear scaling when using large numbers of nodes, which can be seen on the
right side of Figure \ref{fig:total}.
