\subsection*{Representation}
The traditional approach to flocking simulation maintains a matrix
representing the simulated universe which contains birds at specified
indices. This representation is idea for graphics applications, since
the state of the universe is explicitly defined. However, the explicit
representation is less than ideal for scientific reasons--- while
flocking behavior can be simulated correctly and efficiently, the
manner in which the behavior is simulated provides little predictive
power with regard to the cognitive, perceptual, and social mechanisms
required to exhibit flocking behavior. Additionally the explicit
representation has significant memory limitations. Since the entire
state of the universe must be stored, the universe may only be as
large as the total amount of memory available.

In this project we take an agent-based perspective, which favors
computation from the perspective of the individual rather than the
system as a whole. This representation is more cognitively plausible,
since computation must generally follow the steps of each individual
agent. Using the agent-based approach, the universe is represented
\emph{implicitly}. Individual locations are stored locally by the
agent rather than by the individual's index within a global matrix,
meaning that an increase in universe size does not incur additional
memory costs. 

\subsection*{Serial Algorithm}
The standard serial flocking simulation follows a familiar simulation
format. Roughly, the algorithm calculates the three flocking rules for
each bird, stores the result in a temporary field, then later applies
the rules to each bird after all calculations are complete. The
separation of movement calculation and positional upating guarantees
that all movement occurs simultaneously within a time-step.

\begin{verbatim}
  for ( time-steps 0...t ) {
    /* Calculate bird movements */
    for ( Bird b ) {
      N = b's flock neighbors
      alignment = avg. direction over N
      cohesion = avg. x,y,z pos. over N
      separation = -avg. distance to n in N

      calculate dx/dy/dz
    }

    /* Update the bird's positions */
    for ( Bird b ) {
      x += dx
      y += dy
      z += dz
    }
  }
\end{verbatim}

The skeleton of the simulation, as well as functionality for setting
parameters by command-line arguments was implemented by Mitchell
Mellone.  The flocking rules were implemented by Kevin O'Neill based
largely on the original simulation of boid flocking in
\cite{Reynolds}. In contrast, the representation of birds as C structs
was originally implemented by Mitchell Mellone, and later optimized
and extended for 3D by Kevin O'Neill.

\subsection*{Parallel Algorithm}
The parallel implementation of the simulation operates in much the
same way, with the provision that each process (MPI rank) is now
responsible for a subset of birds of the flock. Processes must then
communicate their local information to all other processes in order
for update rules to include all applicable birds. This communication
is implemented using MPI\_Allgather; each rank recieves a complete
array of all birds in the universe to use in computation, but is only
responsible for calculating the position updates for its own agents.
Because MPI\_Allgather is a collective operation, ranks are
synchronized over each time-step, and the simulation is therefore
cycle-accurate.

We extend this approach to include functionality for
multithreading. Where each rank is assigned to a subset of agents,
each constituent thread of a rank is responsible for a subset of
its rank's agents. MPI communication is only performed by the parent
thread, and threads are synchronized between time-steps.

This leads us to another benefit of an implicit representation---
where an explicit representation would be parallelized by assigning
each rank to a sector of the universe, the implicit representation
assigns agents to ranks. Computation is then more evenly distributed
between processes with the implicit representation, since with the
explicit representation entire sectors of the universe could be
empty. However, we acknowledge that this benefit comes at a
substantial cost in communication overhead. Since the neighbors of an
agent can be maintained by any rank, each rank must receive a complete
array of all agents in the universe. Ideally ranks would only need to
gather agents which are relevant to its computation.

The parallelization of the original simulation in MPI (including
necessary communication between ranks) was implemented by Kevin
O'Neill. Afterwards multi-threading was implemented by Mitchell
Mellone using the pthread library.

