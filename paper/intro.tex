Flocks of birds follow synchronized movements which give the
appearance of a uniform being. The motion of the flock is not guided
by any single member, nor is there scientific evidence for substantial
communication or cooperative decision between members.  Rather, the
movement of the flock is determined by implicit guidelines restricting
the motion of each individual. These three guidelines (alignment,
cohesion, and separation) are simple and intuitive, but generate
extreme complexity when applied across massive groups of individuals.

Alignment directs birds in the average direction of neighboring birds
in the flock. Similarly, cohesion directs birds towards the average
location of neighboring birds. Both of these principles ensure that
the flock maintains coherency, so that individual birds are not lost
from the flock. In contrast, separation directs birds away from
neighboring birds which are too close. Separation then creates room
within a flock, so that individual members are not crowded or
co-located \cite{Reynolds}.
