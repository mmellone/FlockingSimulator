We set out in this project with the goal of creating a high performance,
massively parallel simulation of flocking behavior in birds using an implicit
representation of the universe. We found that, through a strong scaling study,
this simulation scales well to many parallel processors, provided that a good
balance of MPI ranks and pthreads are used. In particular, MPI collective
operations scaled very well as the total number of compute nodes increased,
however, sending a large amount of small data packages does not appear to be
as efficient as sending larger packages of data between fewer ranks. A weak
scaling study showed that there is still much room for improvement in our update
algorithm in order to create better performance for very large numbers of birds.

\subsection*{Future Work}
The performance of the position update algorithm can be vastly
improved by using a data structure more efficient than a simple array
to hold the bird data types. In particular we want a data structure
that makes it easier to find birds that are located closely within the
universe. This could be accomplished using a multidimensional binary
search tree, where it has been shown that regional searches can be
done in \(O(kN^{1-(1/k)})\) time \cite{Lee}. This would be a vast
improvement over our current \(O(n^2)\) search implementation.

Additionally we could also modify the MPI message passing to gather
only the birds which are relevant to the MPI rank collecting
them. Although collective message passing is quite efficient, as we
continue to increase the number of simulated agents it becomes
impractical to keep a copy of every agent in each rank and perform a
large MPI\_Allgather each iteration. Instead, by sending birds to each
rank separately and by keeping track of which birds are utilized by
that rank, we could potentially rely on the principle of locality to
reduce the communication overhead of globally sharing every bird to
every rank.

Alternatively, a completely different conception of our approach to
parallelization could generate substantial savings in both
communication overhead and computation time. In our approach, birds
are assigned to an MPI rank statically upon initialization. This
balances the workload between ranks, but unfortunately increases the
total amount of work since every rank must gather birds which are
irrelevant to its computation. By dynamically assigning birds to MPI
ranks, we can associate each cluster or flock of birds with a specific
set of MPI ranks which need only communicate with each other. Then the
only communication necessary between disparate ranks would be for the
dynamic reassignment of birds to a different rank. Since dynamic
assignment reduces the number, size, and breadth of messages, it has
the potential for large performance gains at the cost of a more
complex implementation.

\subsection*{Applications}
While this simulation was specifically designed to imitate flocking
birds, there are many other applications that can make use of a
efficient and highly scalable simulator such as this. For example, the
emergent principles of flocking birds can be applied to create dynamic
visualizations of temporal data evolutions, such as stock
data \cite{Moere}. The simulation of the interactions between many
agents can even be used to select songs on an internet radio based on
what other people are currently listening to \cite{Ibáñez}.

There are many more uses for efficient simulations of emergent flocking behavior.
Our techniques of using an implicitly defined universe and combining
an optimal number of pthreads with MPI ranks provide a good base model that, if
combined with some more advanced data structures and smart message passing, can
be very effective for many of these uses.
\bigskip\\
The code for this project is on Mitchell Mellone's (mellom3) kratos account.
