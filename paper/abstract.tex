Flocking is an emergent behavior found in birds (as well as other
animals) where many individuals in a “flock” move collectively as a
single group. These movements are not governed by any central
coordination, but rather arise from many individuals following simple
rules. In this simulation we focus on three major rules: separation
(avoid very close neighbors), alignment (move in the same direction as
neighbors), and cohesion (move in the direction of the average
position of neighbors). By allocating a subset of individual birds to
a separate process or thread and communicating between processes using
MPI, we show that this simulation can be run efficiently in a
high-performance computing setting. In addition to analyzing the
accuracy of the flocking behavior demonstrated in the simulation
through real-time visualizations, we analyze the strong scalability of
the simulation with respect to process count, strong scalability with
respect to process and thread count, and weak scalability with respect to
increasing node count.
